%% Template for SDP report, adapted from mlp_cw2_template, 2018. 

%% Based on  LaTeX template for ICML 2017 - example_paper.tex at 
%%  https://2017.icml.cc/Conferences/2017/StyleAuthorInstructions

\documentclass{article}
\usepackage[T1]{fontenc}
\usepackage{amssymb,amsmath}
\usepackage{txfonts}
\usepackage{microtype}
\usepackage{xspace}
\xspaceaddexceptions{\%}

% Lists with less spacing between items
\usepackage{paralist}

% For figures
\usepackage{graphicx}
\usepackage{subfig} 

% For citations
\usepackage{natbib}

% For algorithms
\usepackage{algorithm}
\usepackage{algorithmic}

% the hyperref package is used to produce hyperlinks in the
% resulting PDF.  If this breaks your system, please commend out the
% following usepackage line and replace \usepackage{mlp2017} with
% \usepackage[nohyperref]{mlp2017} below.
\usepackage{hyperref}
\usepackage{url}
\urlstyle{same}

% Packages hyperref and algorithmic misbehave sometimes.  We can fix
% this with the following command.
\newcommand{\theHalgorithm}{\arabic{algorithm}}


% Set up MLP coursework style (based on ICML style)
\usepackage{mlp2018}
\mlptitlerunning{SDP Demo \demoNumber  Group (\groupNumber)}
\bibliographystyle{icml2017}


\DeclareMathOperator{\softmax}{softmax}
\DeclareMathOperator{\sigmoid}{sigmoid}
\DeclareMathOperator{\sgn}{sgn}
\DeclareMathOperator{\relu}{relu}
\DeclareMathOperator{\lrelu}{lrelu}
\DeclareMathOperator{\elu}{elu}
\DeclareMathOperator{\selu}{selu}
\DeclareMathOperator{\maxout}{maxout}







%% You probably do not need to change anything above this comment

%% REPLACE the details in the following commands with your details
\setGroupNumber{1}
\setGroupName{My Group}
\setProductName{My Product}
\setStudentName{My Name}
\setLogoFileName{figs/sdp_logo_placeholder.png}

\begin{document} 

\makeSDPTitle{Individual report}

% Previous MLP Style Title Layout working. 
% \twocolumn[
    % \mlptitle{\productName: SDP Demo \demoNumber}
    % \centerline{Group \groupNumber: \groupName}
% ]

\begin{abstract} 
Your individual report should give a clear description of your personal contribution to your project. 
The abstract should first consist of one sentence recalling the main functionality of your system, followed by one or two sentences
stating your major contribution to the project. A few sentences conclude the abstract with a brief description of the main lessons learned during the project. Overall the abstract should not be longer than 200 words.
\end{abstract} 

\section*{Introduction}
\label{sec:intro}
This document provides a template for the SDP individual report.  This template structures the report into sections, which you are required to use. You can change the subsection headings if you wish. In this template the text in each section will include an outline of what you should include in each section, along with some practical LaTeX examples (for example figures, tables, algorithms).  Your document length should be between \textbf{two and 3 pages}. 

You should delete this introduction section (no introduction is required).

\section{Contribution} 
This section should provide clear description of one key contribution that you were directly responsible for (e.g. designing the physical robot, solving a software problem, pulling together several strands of work). This should detail what you did and why it was important (1-2 pages). If possible,
highlight the concrete outcome of the contribution to your project.

\section{Lessons learned}
This section should provide a reflection on what skills you acquired in the course, what you learned about yourself, and what you learned about group work and project management. This can include lessons about what not to do - e.g., you could explain something the group did which turned out to be a mistake - the assessment will be based on your ability to recognise and learn from this, rather than penalising you for having made a mistake! (1 page).

\section*{Submission}
This section is to be deleted.

Each group member should submit their own document on Learn.
The filename must be [g]-[s]-report.pdf where [g] is your group number and  [s] is your student number.

The document should also be emailed to the group mentor at the time of submission.


%% Include any references in a bibliography

\bibliography{example-refs}

\end{document} 

